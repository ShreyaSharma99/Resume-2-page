% ----------------------------------------------------------------------------------------
% SECTION TITLE
% ----------------------------------------------------------------------------------------

\cvsection{Projects}

% ----------------------------------------------------------------------------------------
% SECTION CONTENT
% ----------------------------------------------------------------------------------------

\begin{cventries}

  % ------------------------------------------------
  
  \cventry
  {Undergraduate Thesis Project, Prof. Parag Singla \href{https://smt-comp.github.io/2019/system-descriptions/cvc4-symbreak.pdf}{[Report]}}
  {\href{https://docs.google.com/presentation/d/1TIlfgccQIz8VmogkZZMk7qBsq-PIYC3zyI5vv6IkX9w/edit?usp=sharing}{\entrytitlestyle{Symmetry Breaking for SMT Problems}}
    \ \ \ \normalfont\href{https://docs.google.com/presentation/d/1TIlfgccQIz8VmogkZZMk7qBsq-PIYC3zyI5vv6IkX9w/edit?usp=sharing}
    {}}
  {IIT Delhi}
  {July 2018 - Present}
  {
    \begin{cvitems}
    \item Incorporating symmetries in SAT Modulo Theories(SMT)  framework to exploit syntactical structure of domain 
    \item Detecting symmetries and generating efficient Symmetry Breaking Predicates(SBPs) which reduce search space without affecting satisfiability  
    \item Implemented solver CVC4-SymBreak; participated in SMT-COMP 2019; improved performance over CVC4-2018
    \end{cvitems}
  }
  
  \cventry
  {Course Project, Natural Language Processing, Prof. Mausam \href{http://www.cse.iitd.ac.in/nlpdemo/abst_dial}{[Demo]}
  \href{http://www.cse.iitd.ac.in/nlpdemo/abst_dial/report.pdf}{[Report]}
  }
  {\href{https://github.com/saketdingliwal/Abstractive-Dialogue-Summarization}{\entrytitlestyle{Abstractive Summarization of Dialogues}}
    \ \ \ \normalfont\href{https://github.com/saketdingliwal/Abstractive-Dialogue-Summarization}
    {}}
  {IIT Delhi}
  {March 2018 - May 2018}
  {
    \begin{cvitems}
    \item Used Pointer-Generator Network on top of Attention based model for abstractive summary of dialogues.
    \item Used CRF for sequence labeling of the discourse relations in dialogues to get graph-like structure for them.
    \item Obtained BLEU scores comparable to state of the art news articles summarization tools for the domain of dialogues.
    \end{cvitems}
  }
  
  \cventry
  {Institute Self Driving Project, Prof. Subhashis Banerjee}
  {\href{https://github.com/AniketBajpai/AutoNav2}{\entrytitlestyle{Autonomous Driving in Campus}}
    \ \ \ \normalfont\href{https://github.com/AniketBajpai/AutoNav2}
    {}}
  {Mahindra Rise | IIT Delhi}
  {July 2017 - Dec. 2017}
  {
    \begin{cvitems}
    \item Marked lane boundaries, computed it's curve equations \& stabilized it with SEGNET, hough transform, Kalman Filter.
    \item SLAM Module for obstacle detection, odometry for         vehicle position on road and planned path using RRT         algorithm.
    \item Achieved correct steering angle predictions, lane markers' equations on live feed from car's dashboard Zed camera.
    \end{cvitems}
  }
  
  \cventry
  {Course Project, Computer Vision, Prof. Subhashis Banarjee
  \href {https://github.com/ankesh007/Body-Measurement-using-Computer-Vision/blob/master/Presentation.pdf} {[Report]}}
  {\href{https://github.com/ankesh007/Body-Measurement-using-Computer-Vision}{\entrytitlestyle{Auto Body Measurement using Single Camera}}
    \ \ \ \normalfont\href{https://github.com/ankesh007/Body-Measurement-using-Computer-Vision}
    {}}
  {IIT Delhi}
  {Sept. 2017 - Nov. 2017}
  {
    \begin{cvitems}
    \item Affine \& Metric Correction of image with reference checkerboard object to make lines of measurement parallel.
    \item Grub-Cut Segmentation of image to obtain silhouette of the body to auto detect important measurement points.
    \item Used HAAR cascades and heuristics on the silhouette and achieved correct body measurements within small error.
    \end{cvitems}
  }

  % ------------------------------------------------

 \cventry
  {Winter Undergraduate Project, Prof. Anshul Kumar}
  {\href{https://drive.google.com/file/d/0B7I8cEc5HGiaaDZNNnJESV94T3FiaFp6elZBZEk5Wm9Ub19z/view?usp=sharing}{\entrytitlestyle{Hardware Accelerator for Convolutional Neural Network}}
    \ \ \ \normalfont\href{https://drive.google.com/file/d/0B7I8cEc5HGiaaDZNNnJESV94T3FiaFp6elZBZEk5Wm9Ub19z/view?usp=sharing}
    {}}
  {IIT Delhi}
  {Dec. 2016 - Jan. 2017}
  {
    \begin{cvitems}
    \item Designed hardware approaches using minimal amount of clock cycles,memory and hardware resources.
    \item Implemented a state machine to make pipeline of multiplication \& accumulation through adder tree, block memory.
    \item Coded VHDL programs to implement sliding window of registers for multiplication of image and kernels' matrices.
    \end{cvitems}
  }

\end{cventries}

%%% Local Variables:
%%% mode: xelatex
%%% TeX-master: "../resume_twopage.tex"
%%% End:

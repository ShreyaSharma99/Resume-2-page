% ----------------------------------------------------------------------------------------
% SECTION TITLE
% ----------------------------------------------------------------------------------------

\cvsection{Projects}

% ----------------------------------------------------------------------------------------
% SECTION CONTENT
% ----------------------------------------------------------------------------------------

\begin{cventries}

  % ------------------------------------------------
  
  \cventry
  {Undergraduate Thesis Project, Prof. Parag Singla}
  {\href{https://github.com/saketdingliwal/MLN-Inference-Task}{\entrytitlestyle{Scaling up Markov Logic Network Inference}}
    \ \ \ \normalfont\href{https://github.com/saketdingliwal/MLN-Inference-Task}
    {}}
  {IIT Delhi}
  {July 2018 - Present}
  {
    \begin{cvitems}
    \item Finding independent components in inference of Markov Logic Networks \& implementing parallel Gibbs Sampling.
    \item Exploiting symmetries in functions of first order logic and using Satisfiability Modulo Theroy solvers to formulate inference problem.
    \item Scaling Markov Logic Networks to larger dataset and networks to compare with deep learning models.
    \end{cvitems}
  }

  \cventry
  {Institute Self Driving Project, Prof. Subhashis Banerjee}
  {\href{https://github.com/AniketBajpai/AutoNav2}{\entrytitlestyle{Autonomous Driving in Campus}}
    \ \ \ \normalfont\href{https://github.com/AniketBajpai/AutoNav2}
    {}}
  {IIT Delhi}
  {July 2017 - Dec. 2017}
  {
    \begin{cvitems}
    \item Marked lane boundaries, computed it's curve equations \& stabilized it with SEGNET, hough transform, Kalman Filter.
    \item SLAM Module for obstacle detection, odometry for         vehicle position on road and planned path using RRT         algorithm.
    \item Achieved correct steering angle predictions, lane markers' equations on live feed from car's dashboard Zed camera.
    \end{cvitems}
  }

  \cventry
  {Course Project, Natural Language Processing, Prof. Mausam}
  {\href{https://github.com/saketdingliwal/Abstractive-Dialogue-Summarization}{\entrytitlestyle{Abstractive Summarization of Dialogues}}
    \ \ \ \normalfont\href{https://github.com/saketdingliwal/Abstractive-Dialogue-Summarization}
    {}}
  {IIT Delhi}
  {March 2018 - May 2018}
  {
    \begin{cvitems}
    \item Used Pointer-Generator Network on top of Attention based model for abstractive summary of dialogues.
    \item Used CRF for sequence labeling of the discourse relations in dialogues to get graph-like structure for them.
    \item Obtained BLEU scores comparable to state of the art news articles summarization tools for the domain of dialogues.
    \end{cvitems}
  }
  
  \cventry
  {Course Project, Computer Vision, Prof. Subhashis Banarjee}
  {\href{https://github.com/saketdingliwal/Abstractive-Dialogue-Summarization}{\entrytitlestyle{Abstractive Summarization of Dialogues}}
    \ \ \ \normalfont\href{https://github.com/saketdingliwal/Abstractive-Dialogue-Summarization}
    {}}
  {IIT Delhi}
  {March 2018 - May 2018}
  {
    \begin{cvitems}
    \item Used Pointer-Generator Network on top of Attention based model for abstractive summary of dialogues.
    \item Used CRF for sequence labeling of the discourse relations in dialogues to get graph-like structure for them.
    \item Obtained BLEU scores comparable to state of the art news articles summarization tools for the domain of dialogues.
    \end{cvitems}
  }

  % ------------------------------------------------
\end{cventries}

%%% Local Variables:
%%% mode: xelatex
%%% TeX-master: "../resume_twopage.tex"
%%% End:
